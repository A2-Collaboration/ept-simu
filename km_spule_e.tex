\newpage

{\Large \bf Specification of coils for a correction magnet\\[2ex]}
At the Institute for Nuclear Physics at the Mainz University (Germany)
we run the electron accelerator MAMI (Mainz Microtron).
For an application in the beam transport between an endpoint-tagger,
ET, and
the main-tagger, MT, we need a correction magnet, KM (see drawings). 
This magnet has a pole
gap of 20\,mm and should reach at least 1.4\,T.
The iron parts can be machined in our own workshop, for the coils of the KM
we herewith ask for a bid.

We have a power supply (120\,A, 36\,V) which we would like to use in this
context.

The correction magnet in principle has the form of a C-magnet, 
such that the
inner dimensions of the coils follow the shape of the pole piece plus
a gap of 5\,mm.
Thus, the coils are two identical parts each of them should have
34000 ampere turns. The connections for the electricity and for the cooling
liquid (demineralized water) shall be at the longer side of the coils. 
At the ends of the coils zinc-free bronze couplings are needed which do not
act as a resistance for the water flow through the copper conductors.
At the ends of the conductors tinned copper connection plates are to be
mounted. The coils need to be wound in one-layer winding technique, such
that a homogeneous temperature distribution can be achieved. At a
water-pressure difference of 6 bar the increase of temperature between
in- and outlet should not surmount 20\,K. According to the drawings the
dimension of the coils should be within 
90\,mm $\times$ 190\,mm. If one of these can be decreased it should
preferably be the smaller one, which would allow to get the three
components (ET,KM and MT) closer to each other.

The KM will be bolted to yoke-parts of ET, the final design will be made
after knowing the dimensions of the coils. The size of the coils will not
require weldings or soldering inside the packages.

Quality control can be negotiated.


