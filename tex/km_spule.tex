
{\Large \bf Spezifikation von Spulen f\"ur einen Korrektur-Magneten\\[2ex]}
Am Instituit f\"ur Kernphysik der Universit\"at Mainz wird der
Elektronenbeschleuniger "Mainzer Mikrotron (MAMI)" betrieben.
F\"ur eine Anwendung im Strahltransport zwischen einem Endpoint-Tagger,
ET, und dem
Haupt-Tagger, MT, ben\"otigen wir einen Korrektur-Magneten, KM,
mit einem Pol\-schuh\-ab\-stand von 20\,mm und einer l\"anglichen, einfachen
Polschuhform (siehe Zeichnungen). Die Eisenteile werden wir in 
eigener Werkstatt
bearbeiten k\"onnen, f\"ur die Spulen des Korrektur-Magneten 
erfragen wir hiermit ein Angebot.

Wir haben ein Netzger\"at (120\,A und 36\,Volt), das wir hierf\"ur,
wenn m\"oglich, einsetzen m\"ochten. 

Der Korrektur-Magnet wird die Grundform eines C-Magneten
haben, so dass die Innenabmessung der Spulen mit einer Tolerenz von etwa 
5\,mm der Polschuhform entsprechen. Die Gesmatspule besteht also aus zwei
Teilspulen, von denen jede etwa 34000 Amp\`{e}re-Windungen 
haben sollte, damit
wir im Korrektur-Magneten mindestens 1.4\,T er\-rei\-chen.
Strom- und K\"uhlmittelanschl\"usse sollen an der l\"angeren Seite liegen.
An den Enden eines jeden Stromleiters sind Schlauchwellen aus zinkfreier
Bronze vorzusehen, diese Schlauchwellen d\"urfen keine Verengung des
Durchflussquerschnitts im Kupferleieter darstellen. An den Spulenenden
sind passende Anschlussplatten (Cu verzinnt) anzubringen. Die Spulen sollen
in einlagiger Wickeltechnik erstellt werden, um eine gleichm\"a{\ss}ige
Temperaturverteilung zu erhalten. Bei einer Druckdifferenz von 6 bar
zwischen Vor- und R\"ucklauf sollte die Temperaturerh\"ohung des K\"uhlmittels
(deminearlisiertes Wasser)
20\,K nicht \"uberschreiten. Der Querschnitt der Spulen sollte 
90\,mm $\times$ 190\,mm (letztere in Richtung des Feldes) 
nicht \"uberschreiten,
wobei m\"ogliche Platz\-ein\-spa\-run\-gen bei den 90\,mm 
erw\"unscht sind, damit
die drei optischen Komponenten ET, KM und MT m\"oglichst raumsparend
aufgestellt werden k\"onnen.

Der Korrektur-Magnet wird mit dem Joch des vorhandenen ET-Magneten 
verschraubt, die 
endg\"ultige Planung hierf\"ur wird nach Kenntnis der Spulendimensionen
durchgef\"uhrt. Die Gr\"o{\ss}e der Spulen wird es nicht erforderlich machen,
L\"otstellen innerhalb des Spulenpakets zu haben.

Zur Qualit\"atspr\"ufung k\"onnen Vereinbarungen getroffen werden.


